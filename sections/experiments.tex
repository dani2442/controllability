\section{Numerical Experiments}
\label{sec:experiments}

We validate the theoretical results through numerical experiments implemented in PyTorch.
All simulations use Euler--Maruyama time stepping to integrate controlled linear SDEs of the form
\[
  \d x(t) = (\mat{A}x(t) + \mat{B}u(t))\,\d t + \beta\,\d W(t),
\]
where $\mat{A}\in\R^{n\times n}$ is the drift matrix, $\mat{B}\in\R^{n\times m}$ the input matrix, $\beta\in\R^{n\times q}$ the noise intensity, and $W$ a $q$-dimensional standard Brownian motion.
Unless stated otherwise, the input is a deterministic multi-sinusoid with each channel given by
\[
  u_i(t)=\sin(\omega_i t),\qquad i=1,\dots,m,
\]
where $\omega_1,\ldots,\omega_m\in\mathbb{R}$ are distinct frequencies chosen to avoid aliasing.

\subsection{Validation of the Data-Driven Hautus Test}
\label{subsec:exp-hautus}

Our first experiment validates the data-driven Hautus test by comparing the estimated matrix $\widehat{\mat{P}}_\lambda$ with the true Hautus matrix $\mat{P}_\lambda = [\mat{A}-\lambda\mat{I}, \, \mat{B}]$.
We simulate a system with $n=6$ states, $m=3$ inputs, and $q=2$ noise dimensions over a horizon $T=100$ with time step $\d t = 0.05$.
The drift matrix $\mat{A}$ is constructed to be Hurwitz stable, with all eigenvalues having real parts bounded by~$-0.1$.

For each test value $\lambda\in\CC$, we compute both time-domain and FFT-based estimates via the cross-moment formulation~\eqref{eq:hautus-cross-moment-P-hat}.

Figure~\ref{fig:hautus-matrix-comparison} shows the element-wise comparison between the true and estimated matrices for a representative $\lambda = 0.3 + 1.3\ii$.
Both methods yield accurate estimates, with estimation errors $\|\widehat{\mat{P}}_\lambda - \mat{P}_\lambda\|_2$ on the order of $10^{-2}$.
To assess variability, we run $50$ independent trajectories and examine the distribution of errors across the eigenvalues of $\mat{A}$ and additional random test points.
Figure~\ref{fig:hautus-error-summary} (left) summarizes the signed error $\sigma_{\min}(\mat{P}_\lambda) - \sigma_{\min}(\widehat{\mat{P}}_\lambda)$, showing near-zero bias and comparable variance for time-domain and FFT estimators.

\begin{figure}[t]
  \centering
  \includegraphics[width=0.95\textwidth]{images/hautus_matrix_comparison.pdf}
  \caption{Element-wise comparison of the true Hautus matrix $\mat{P}_\lambda$ (left) and its time-domain estimate $\widehat{\mat{P}}_\lambda$ (center), with absolute difference (right). The small difference magnitudes confirm the accuracy of the cross-moment estimator.}
  \label{fig:hautus-matrix-comparison}
\end{figure}

\subsection{Error Convergence Rate}
\label{subsec:exp-convergence}

Proposition~\ref{prop:hautus-cross-moment-high-prob} predicts that the estimation error $\|\widehat{\mat{P}}_\lambda - \mat{P}_\lambda\|_2$ decays at the rate $\mathcal{O}(T^{-1/2})$ as the observation horizon~$T$ increases.
We validate this theoretical prediction by simulating systems with $n=5$ states, $m=3$ inputs, and $q=2$ noise dimensions across horizon lengths $T \in \{10, 20, 50, 100, 200, 500, 1000\}$.
For each value of~$T$, we run $50$ Monte Carlo trials with independent noise realizations and random test eigenvalues~$\lambda$.

Figure~\ref{fig:hautus-error-summary} (right) presents the results on a log-log scale.
The empirical mean errors for both time-domain and FFT methods closely follow the $T^{-1/2}$ rate.
The shaded region indicates $\pm 1$ standard deviation across trials.
We also overlay the theoretical bound from Proposition~\ref{prop:hautus-cross-moment-high-prob} (computed with $\delta = 0.05$), which provides a valid upper envelope for the empirical errors.
A linear regression on the log-transformed data yields an estimated slope of approximately $-0.5$, confirming the theoretical rate.

\begin{figure}[t]
  \centering
  \begin{subfigure}[t]{0.49\textwidth}
    \centering
    \includegraphics[width=\textwidth]{images/hautus_error_violinplot.pdf}
    \caption{Error distribution across $50$ trajectories (eigenvalues of $\mat{A}$ and random test points).}
    \label{fig:hautus-violinplot}
  \end{subfigure}\hfill
  \begin{subfigure}[t]{0.49\textwidth}
    \centering
    \includegraphics[width=\textwidth]{images/error_convergence.pdf}
    \caption{Mean error versus horizon $T$ (log-log) with $\pm1$ std shading and the theoretical bound.}
    \label{fig:error-convergence}
  \end{subfigure}
  \caption{Summary of estimation accuracy: variability across trials (left) and $T^{-1/2}$ convergence (right).}
  \label{fig:hautus-error-summary}
\end{figure}

\subsection{Finite Candidate Eigenvalue Check}
\label{subsec:exp-finite-candidates}

A key practical advantage of the cross-moment formulation is that controllability can be verified by testing only a finite set of candidate eigenvalues rather than all $\lambda\in\CC$.
Specifically, the operator formulation developed in Section~\ref{subsec:hautus-operator} shows that rank deficiency of the pencil $\mathcal{P}(\lambda)=\dot\X-\lambda\X$ can only occur when $\lambda$ belongs to the spectrum of the data-driven matrix
\[
  \mat{K} := \mat{S}_X^{-1}\mat{M}_X,
  \quad\text{where}\quad
  \mat{S}_X := \int_0^T x(t)x(t)^\top\,\d t,
  \quad
  \mat{M}_X := \int_0^T x(t)\,\d x(t)^\top.
\]

We simulate a controllable system with $n=6$, $m=3$, $q=2$ over a horizon $T=100$.
The candidate eigenvalues $\sigma(\mat{K})$ are computed from the trajectory data and compared with the true eigenvalues $\sigma(\mat{A})$.
For each candidate $\lambda_i$, we compute the estimated minimum singular value $\sigma_{\min}(\widehat{\mat{P}}_{\lambda_i})$ and compare with the true value $\sigma_{\min}(\mat{P}_{\lambda_i})$.

The results show excellent agreement between estimated and true singular values across all candidates.
The candidates $\sigma(\mat{K})$ cluster near the true eigenvalues $\sigma(\mat{A})$ in the complex plane, validating the theoretical prediction that the data-driven eigenvalue estimates converge to the true system eigenvalues.
Since $\sigma_{\min}(\widehat{\mat{P}}_\lambda) > 0$ for all candidates, the system is correctly identified as controllable.

\subsection{Comparison of Time-Domain and FFT Methods}
\label{subsec:exp-method-comparison}

We compare two computational approaches for evaluating the cross-moments: (i)~the \emph{time-domain method}, which computes $\mat{H}_\lambda(T)$ via direct numerical integration, and (ii)~the \emph{FFT method}, which leverages the Parseval identity with frequency-domain representations as in~\eqref{eq:hautus-fourier-y}.

For a system with $n=8$ states, $m=4$ inputs, $q=2$ noise dimensions, and horizon $T=100$ with time step $\d t = 0.02$ (yielding $5001$ sample points), we test both methods across various values of~$\lambda$.
Both methods achieve comparable accuracy, with estimation errors typically differing by less than $10\%$.

The FFT method requires a frequency cutoff $\omega_{\max}$ to truncate the integral.
Figure~\ref{fig:omega-cutoff-table} shows the effect of $\omega_{\max}$ on the estimation error for $\lambda = 0.3 + 0.8\ii$.
As expected, larger cutoffs improve accuracy, with diminishing returns beyond $\omega_{\max} \approx 100$.
For compactness, we place the cutoff summary next to a representative 3D phase portrait (Figure~\ref{fig:trajectory-3d}).

\begin{figure}[t]
  \centering
  \begin{subfigure}[b]{0.48\textwidth}
    \centering
    \vspace{0pt}
    \renewcommand{\arraystretch}{1.1}
    \begin{tabular}{@{}ccc@{}}
      \toprule
      $\omega_{\max}$ & $\sigma_{\min}(\widehat{\mat{P}}_\lambda)$ & $\|\widehat{\mat{P}}_\lambda - \mat{P}_\lambda\|_2$ \\
      \midrule
      10 & 0.412 & 0.089 \\
      20 & 0.428 & 0.051 \\
      50 & 0.435 & 0.027 \\
      100 & 0.438 & 0.018 \\
      200 & 0.439 & 0.015 \\
      $\infty$ (all) & 0.439 & 0.014 \\
      \bottomrule
    \end{tabular}
    \caption{Effect of frequency cutoff $\omega_{\max}$ on FFT method accuracy for $\lambda = 0.3 + 0.8\ii$.}
    \label{fig:omega-cutoff-table}
  \end{subfigure}\hfill
  \begin{subfigure}[b]{0.48\textwidth}
    \centering
    \vspace{0pt}
    \includegraphics[width=\textwidth]{images/trajectory_3d.pdf}
    \caption{Three-dimensional phase portraits for $(x_0,x_1,x_2)$, $(x_3,x_4,x_5)$, and $(x_6,x_7,x_8)$.}
    \label{fig:trajectory-3d}
  \end{subfigure}
  \caption{Frequency cutoff sensitivity and trajectory visualization. Left: accuracy of the FFT-based estimator improves with larger cutoff values, with diminishing returns beyond $\omega_{\max}\approx 100$. Right: representative phase portraits for a 9-dimensional controlled stochastic system.}
  \label{fig:method-comparison}
\end{figure}

In terms of computational efficiency, the FFT method offers modest speedups for long trajectories due to the $\mathcal{O}(N\log N)$ complexity of the FFT versus $\mathcal{O}(N)$ for direct integration, though both are fast in practice.

\subsection{Trajectory Visualization}
\label{subsec:exp-trajectory}

To provide intuition for the controlled stochastic dynamics, Figure~\ref{fig:trajectory-3d} displays representative phase portraits for a 9-dimensional system with $m=4$ inputs and $q=2$ noise dimensions, simulated over a horizon $T=50$.
The drift matrix $\mat{A}$ has eigenvalues with real parts near~$-0.05$, corresponding to slowly decaying modes.

The phase portraits demonstrate that the sinusoidal input, combined with the stochastic forcing, provides sufficient excitation to explore the state space.
This persistent excitation ensures that the stacked Gramian $\mat{S}_Z(u)$ is well-conditioned, which is crucial for accurate estimation of the Hautus matrix and reliable controllability certification.
