\section{Numerical Experiments}
\label{sec:experiments}

We validate the theoretical results with numerical experiments implemented in PyTorch.
We simulate controlled linear SDEs of the form
\[
  \d x(t) = (\mat{A}x(t) + \mat{B}u(t))\,\d t + \beta\,\d W(t),
\]
using Euler--Maruyama time stepping.

\subsection{Validation of the Data-Driven Hautus Test}
\label{subsec:exp-hautus}

Our first experiment validates the data-driven Hautus test by comparing the estimated matrix $\widehat{\mat{P}}_\lambda$ with the true Hautus matrix $\mat{P}_\lambda = [\mat{A}-\lambda\mat{I} \mid \mat{B}]$.
We simulate a system with $n=6$ states, $m=3$ inputs, and $q=2$ noise dimensions over a horizon $T=100$ with time step $\d t = 0.05$.
The drift matrix $\mat{A}$ is constructed to be stable with eigenvalues having real parts bounded by $-0.1$.

For each $\lambda\in\CC$, we compute both time-domain and FFT-based estimates via the cross-moment formulation \eqref{eq:hautus-cross-moment-P-hat}.

Figure~\ref{fig:hautus-matrix-comparison} shows the element-wise comparison between the true and estimated matrices for a representative $\lambda = 0.3 + 1.3\ii$.
Both methods yield accurate estimates, with estimation errors $\|\widehat{\mat{P}}_\lambda - \mat{P}_\lambda\|_2$ on the order of $10^{-2}$.
To assess variability, we run $50$ independent trajectories and examine the distribution of errors across the eigenvalues of $\mat{A}$ and additional random test points.
Figure~\ref{fig:hautus-error-summary} (left) summarizes the signed error $\sigma_{\min}(\mat{P}_\lambda) - \sigma_{\min}(\widehat{\mat{P}}_\lambda)$, showing near-zero bias and comparable variance for time-domain and FFT estimators.

\begin{figure}[t]
  \centering
  \includegraphics[width=0.95\textwidth]{images/hautus_matrix_comparison_time.pdf}
  \caption{Element-wise comparison of the true Hautus matrix $\mat{P}_\lambda$ (left) and its time-domain estimate $\widehat{\mat{P}}_\lambda$ (center), with absolute difference (right). The small difference magnitudes confirm the accuracy of the cross-moment estimator.}
  \label{fig:hautus-matrix-comparison}
\end{figure}

\subsection{Error Convergence Rate}
\label{subsec:exp-convergence}

Proposition~\ref{prop:hautus-cross-moment-high-prob} predicts that the estimation error $\|\widehat{\mat{P}}_\lambda - \mat{P}_\lambda\|_2$ decays as $\mathcal{O}(T^{-1/2})$.
We validate this rate by simulating systems with $n=5$, $m=3$, $q=2$ for horizon lengths $T \in \{10, 20, 50, 100, 200, 500, 1000\}$.
For each $T$, we run $50$ Monte Carlo trials with independent noise realizations and random test eigenvalues $\lambda$.

Figure~\ref{fig:hautus-error-summary} (right) presents the results on a log-log scale.
The empirical mean errors for both time-domain and FFT methods closely follow the $T^{-1/2}$ rate.
The shaded region indicates $\pm 1$ standard deviation across trials.
We also overlay the theoretical bound from Proposition~\ref{prop:hautus-cross-moment-high-prob} (computed with $\delta = 0.05$), which provides a valid upper envelope for the empirical errors.
A linear regression on the log-transformed data yields an estimated slope of approximately $-0.5$, confirming the theoretical rate.

\begin{figure}[t]
  \centering
  \begin{subfigure}[t]{0.49\textwidth}
    \centering
    \includegraphics[width=\textwidth]{images/hautus_error_violinplot.pdf}
    \caption{Error distribution across $50$ trajectories (eigenvalues of $\mat{A}$ and random test points).}
    \label{fig:hautus-violinplot}
  \end{subfigure}\hfill
  \begin{subfigure}[t]{0.49\textwidth}
    \centering
    \includegraphics[width=\textwidth]{images/error_convergence.pdf}
    \caption{Mean error versus horizon $T$ (log-log) with $\pm1$ std shading and the theoretical bound.}
    \label{fig:error-convergence}
  \end{subfigure}
  \caption{Summary of estimation accuracy: variability across trials (left) and $T^{-1/2}$ convergence (right).}
  \label{fig:hautus-error-summary}
\end{figure}

\subsection{Finite Candidate Eigenvalue Check}
\label{subsec:exp-finite-candidates}

A key practical advantage of the cross-moment formulation is that controllability can be verified by testing only a finite set of candidate eigenvalues rather than all $\lambda\in\CC$.
Specifically, the operator formulation (Section~\ref{subsec:hautus-operator}) shows that the candidates are the eigenvalues of the matrix $\mat{K} = \mat{S}_X^{-1}\mat{M}_X$, where $\mat{S}_X = \int_0^T x(t)x(t)^\top\,\d t$ and $\mat{M}_X = \int_0^T x(t)\,\d x(t)^\top$.

We simulate a controllable system with $n=6$, $m=3$, $q=2$ over $T=100$.
The candidate eigenvalues $\sigma(\mat{K})$ are computed from the trajectory data and compared with the true eigenvalues $\sigma(\mat{A})$.
For each candidate $\lambda_i$, we compute the estimated minimum singular value $\sigma_{\min}(\widehat{\mat{P}}_{\lambda_i})$ and compare with the true value $\sigma_{\min}(\mat{P}_{\lambda_i})$.

The results show excellent agreement between estimated and true singular values across all candidates.
The candidates $\sigma(\mat{K})$ cluster near the true eigenvalues $\sigma(\mat{A})$ in the complex plane, validating the theoretical prediction that the data-driven eigenvalue estimates converge to the true system eigenvalues.
Since $\sigma_{\min}(\widehat{\mat{P}}_\lambda) > 0$ for all candidates, the system is correctly identified as controllable.

\subsection{Comparison of Time-Domain and FFT Methods}
\label{subsec:exp-method-comparison}

We compare the time-domain and FFT-based computation methods for the cross-moments.
The time-domain method computes $\mat{H}_\lambda(T)$ via direct numerical integration, while the FFT method uses the Parseval identity with frequency-domain representations \eqref{eq:hautus-fourier-y}.

For a system with $n=8$, $m=4$, $q=2$ and $T=100$ with fine time step $\d t = 0.02$ (yielding $5001$ time points), we test both methods across various $\lambda$ values.
Both methods achieve comparable accuracy, with estimation errors typically differing by less than $10\%$.

The FFT method requires a frequency cutoff $\omega_{\max}$ to truncate the integral.
Table~\ref{tab:omega-cutoff} shows the effect of $\omega_{\max}$ on the estimation error for $\lambda = 0.3 + 0.8\ii$.
As expected, larger cutoffs improve accuracy, with diminishing returns beyond $\omega_{\max} \approx 100$.
For compactness, we place Table~\ref{tab:omega-cutoff} next to a representative 3D phase portrait (Figure~\ref{fig:trajectory-3d}).

\begin{figure}[t]
  \centering
  \begin{minipage}[t]{0.49\textwidth}
    \centering
    \captionsetup{type=table}
    \begin{tabular}{ccc}
      \toprule
      $\omega_{\max}$ & $\sigma_{\min}(\widehat{\mat{P}}_\lambda)$ & $\|\widehat{\mat{P}}_\lambda - \mat{P}_\lambda\|_2$ \\
      \midrule
      10 & 0.412 & 0.089 \\
      20 & 0.428 & 0.051 \\
      50 & 0.435 & 0.027 \\
      100 & 0.438 & 0.018 \\
      200 & 0.439 & 0.015 \\
      $\infty$ (all) & 0.439 & 0.014 \\
      \bottomrule
    \end{tabular}
    \caption{Effect of frequency cutoff on FFT method accuracy.}\label{tab:omega-cutoff}
  \end{minipage}\hfill
  \begin{minipage}[t]{0.49\textwidth}
    \centering
    \captionsetup{type=figure}
    \includegraphics[width=\textwidth]{images/trajectory_3d.pdf}
    \caption{Three-dimensional phase portraits for $(x_0,x_1,x_2)$, $(x_3,x_4,x_5)$, and $(x_6,x_7,x_8)$.}
    \label{fig:trajectory-3d}
  \end{minipage}
\end{figure}

In terms of computational efficiency, the FFT method offers modest speedups for long trajectories due to the $\mathcal{O}(N\log N)$ complexity of the FFT versus $\mathcal{O}(N)$ for direct integration, though both are fast in practice.

\subsection{Trajectory Visualization}
\label{subsec:exp-trajectory}

To provide intuition for the controlled stochastic dynamics, Figure~\ref{fig:trajectory-3d} shows representative phase portraits for a 9-dimensional system with $m=4$ inputs and $q=2$ noise dimensions, simulated over $T=50$ with a slowly decaying drift ($\mat{A}$ has eigenvalues with real parts near $-0.05$).

The phase portraits demonstrate that the stochastic input provides sufficient excitation to explore the state space, ensuring that the signal Gramian $\mat{S}_Z(u)$ is well-conditioned.
This persistent excitation is crucial for accurate estimation of the Hautus matrix and reliable controllability certification.
