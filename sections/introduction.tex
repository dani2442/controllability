\section{Introduction}\label{sec:intro}

\noindent\textbf{Motivation: learning models that are \emph{control-relevant}.}
Modern system identification and data-driven control rely on experiments that generate \emph{informative} data, so that the learned model supports reliable closed-loop guarantees.
In particular, for linear systems, a fundamental prerequisite for many synthesis and certification tasks is that the data reveals the system's \emph{controllable subspace} and, ideally, certifies controllability.
However, learning objectives that focus only on trajectory matching (prediction or simulation error) do \emph{not} automatically preserve control-theoretic properties such as controllability or stabilizability, since distinct systems can generate similar trajectories on a finite horizon while having different reachable dynamics.
This gap motivates experiment design criteria that directly encode control-relevant identifiability.

\noindent\textbf{Persistent excitation and universality in data-driven control.}
The classical notion of \emph{persistent excitation} formalizes the idea that the input must excite all relevant degrees of freedom so that system parameters (or behaviors) become identifiable \cite{willemsNotePersistencyExcitation2005}.
In the behavioral framework, Willems' Fundamental Lemma states that, for controllable LTI systems, all finite-length trajectories can be parameterized from a single measured trajectory provided the input is persistently exciting of sufficiently high order \cite{willemsNotePersistencyExcitation2005}.
This principle underpins a large class of methods for data-driven simulation and control and has been extended in several directions, including multiple datasets and rank-based characterizations \cite{waardeWillemsFundamentalLemma2020,vanwaardeDataInformativityNew2020}.
Recent developments also highlight a converse viewpoint: if one seeks a single experiment design that works \emph{uniformly} for broad classes of controllable systems (``universal'' inputs), then persistent excitation is not merely sufficient but essentially necessary at the right order \cite{shakouriNewPerspectiveWillems2025}.
For continuous-time systems, related identifiability conditions and continuous-time variants of Willems-type results have also been investigated \cite{rapisardaPersistencyExcitationCondition2023,rapisardaFundamentalLemmaContinuoustime2023}.

\noindent\textbf{Offline vs.\ online experiment design.}
Most classical guidance concerns \emph{offline} experiment design, where one selects a fixed open-loop excitation signal before collecting data.
Offline designs are attractive because they are universal and simple to implement, but they can be conservative and sample-inefficient.
This has motivated \emph{online} (adaptive) experiment design methods that adjust the input in real time based on the observed data to accelerate identifiability and reduce the amount of data required \cite{vanwaardePersistentExcitationOnline2022,gramlichFastIdentificationStabilization2022}.
In fact, sharp sample-efficiency results have recently been obtained for the length of an informative trajectory needed for linear system identification \cite{camlibelShortestExperimentLinear2025}.
These developments emphasize that experiment design should be treated as a first-class component of safe data-driven control, rather than a preprocessing step.

\noindent\textbf{Controllability certification from data.}
Alongside trajectory-based learning, there is a growing literature on \emph{control-theoretic tests from data}, including controllability/stabilizability certificates that avoid explicit system identification.
In discrete time, informativity-based tests provide purely data-dependent rank conditions for control properties \cite{vanwaardeDataInformativityNew2020,iannelliDesignInputDatadriven2021}.
More recently, Mishra et al.\ developed algebraic data-driven tests for controllability of LTI systems from batches of measurements \cite{mishraDataDrivenTestsControllability2021}.
These results collectively suggest that controllability can, in principle, be certified directly from suitably designed experiments, even when the system matrices are unknown.

\noindent\textbf{Scope of this paper.}
In this work we revisit experiment design for LTI systems from the perspective of certifying the property of \emph{controllability}.
Consider an unknown continuous-time LTI system with state matrix $\mat A\in\R^{n\times n}$ and input matrix $\mat B\in\R^{n\times m}$, governed by
\begin{equation}\label{eq:lti-intro}
  \dot x(t)=\mat A x(t)+\mat B u(t),\qquad x(0)=x_0.
\end{equation}
The controllable subspace is characterized by the Kalman controllability matrix
\[
  \mat{C}:=[\mat{B},\mat{A}\mat{B},\dots,\mat{A}^{n-1}\mat{B}]\in\R^{n\times nm},
\]
and $\operatorname{rank}(\mat{C})$ determines the dimension of the reachable set from the origin \cite{kalmanMathematicalDescriptionLinear1963a,chenLinearSystemTheory1999}.
An equivalent characterization is given in terms of the classical Hautus (Popov--Belevitch--Hautus, PBH) test \cite[Theorem~3.13]{trentelmanControlTheoryLinear2001}, which characterizes controllability and stabilizability as follows:
\begin{equation}\label{eq:hautus-classical}
\begin{aligned}
(\mat{A},\mat{B}) \text{ is controllable }\quad &\Leftrightarrow \quad \operatorname{rank}[\mat{A}-\lambda \mat{I},\, \mat{B}]=n \quad \forall \lambda\in\CC,\\
(\mat{A},\mat{B}) \text{ \ is stabilizable } \quad &\Leftrightarrow \quad \operatorname{rank}[\mat{A}-\lambda \mat{I},\, \mat{B}]=n \quad \forall \lambda\in\CC \text{ with } \Re(\lambda)\ge 0.
\end{aligned}
\end{equation}
The data-driven Hautus test estimates the rank of the matrix using the following \textit{trick}:
$$
[\mat{A}-\lambda \mat{I},\, \mat{B}]\begin{bmatrix}x(t)\\u(t)\end{bmatrix} = \dot{x}(t) - \lambda x(t)
$$
Therefore, assuming access to $u(\cdot)$ and $x(\cdot)$ on a finite horizon $[0,T]$, one can estimate the rank of the Hautus matrix from data without explicitly identifying $(\mat{A},\mat{B})$. To recover the rank property \textit{from data}, one can form the cross-moment, defined
\[
  \mat H_\lambda(u):=\int_0^T(\dot x(t)-\lambda x(t))z(t)^*\,\d t = [\mat{A}-\lambda \mat{I},\, \mat{B}] \underbrace{\int_0^T z(t)z(t)^* dt}_{\mat{S}_Z(T)} \in \mathbb{C}^{n\times (n+m)},
\]
where $z(t):=[x(t);u(t)]\in\R^{n+m}$ is the stacked state-input vector. The idea is that, if the Gramian $\mat{S}_Z(T)$ is well-conditioned, then the rank of $\mat H_\lambda(u)$ reveals the rank of the Hautus matrix. However, the quality of this estimate depends on the choice of input $u$. 
So, the main question we address in this paper is: \emph{how to design inputs $u$ that reliably reveal the controllability rank from data?}
\subsection{Main Contributions}
\begin{itemize}
  \item \textbf{A continuous-time data-driven Hautus margin.}
  We provide a \textit{continuous-time} and \textit{data-driven} analogue of the Hautus test \eqref{eq:hautus-classical}:
  \begin{equation}\label{eq:hautus-controllable}
    (\mat{A},\mat{B}) \text{ is controllable}\quad\Leftrightarrow\quad
    \operatorname{rank}(\mat{H}_\lambda(u))=n\quad \forall\lambda\in\CC.
  \end{equation}
  assuming $\mat{S}_Z(T)$ is well-conditioned. It suffices to check a finite set $\lambda \in \sigma(\mat{K})$, where
  \[
    \mat{K}:=\Big(\int_0^T x(t)x(t)^*\,\d t\Big)^{-1}\Big(\int_0^T x(t)\dot x(t)^*\,\d t\Big).
  \]
  We further develop a derivative-free formulation using Fourier transform.

  \item \textbf{Input design for conditioning the Hautus margin.}
  We provide sharp, model-agnostic designs for conditioning the input, which improves the data-driven test \eqref{eq:hautus-controllable}:\\
  (1) Under an $L^2$ energy budget $\|u\|_{L^2}\le 1$, the best possible conditioning is achieved by any input with $m$ orthonormal time functions (Proposition~\ref{prop:hautus-isotropic-input}), yielding
  $$\int_0^T u(t)u(t)^\top\,\d t=\frac{1}{m}\mat{I}_m.$$
  (2) Under an $H^1$ smoothness budget $\|u\|_{H^1}\le 1$, the \textit{unique} optimal design is given by
  \[
    u(t) = \sqrt{\alpha}\,Q\begin{bmatrix}\psi_0(t)\\ \vdots\\ \psi_{m-1}(t)\end{bmatrix},
    \quad\text{where}\quad
    \psi_k(t):=\sqrt{\tfrac{2}{T}}\cos\!\Big(\frac{k\pi t}{T}\Big),
  \]
  with $Q\in\R^{m\times m}$ an orthogonal matrix and $\alpha$ a normalization term (Proposition~\ref{prop:hautus-isotropic-input-h1}).

  \item \textbf{Concentration bounds under It\^o noise.} Consider the It\^o noise model
  \[
    \d x(t) = \big(\mat{A}x(t) + \mat{B}u(t)\big)\,\d t + \beta\,\d W(t),
  \]
  where $W$ is a $q$-dimensional standard Brownian motion and $\beta\in\R^{n\times q}$ is the noise intensity matrix.
  In Proposition~\ref{prop:hautus-cross-moment-high-prob}, we prove that with probability at least $1-\delta$,
  \[
    \|\widehat{\mat{P}}_\lambda(T)-\mat{P}_\lambda\|_2
    \le
    \frac{\|\beta\|_2}{\sqrt{T\,\sigma_{\min}(\bar{\mat{S}}_Z(T))}}
    \Big(\sqrt{q}+\sqrt{n+m}+\sqrt{2\log(1/\delta)}\Big) = \mathcal{O}\Big(\frac{1}{\sqrt{T}}\Big),
  \]
  where $\mat{P}_\lambda:=[\mat{A}-\lambda \mat{I}, \, \mat{B}]$, $\hat{\mat{P}}_\lambda(T)$ is the data-driven estimate and $\bar{\mat{S}}_Z(T):=\frac{1}{T}\int_0^T z(t)z(t)^\top\,\d t$ is its normalized Gramian.
\end{itemize}
Together, these results provide a principled experiment-design toolkit for learning models that preserve controllability structure and are therefore suitable for safe downstream control.


