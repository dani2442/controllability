
\section{Stochastic model and It\^o residual}
\label{sec:stochastic}

The deterministic framework of Section~\ref{sec:problem} assumes access to $\dot x$ or relies on forming derivatives from sampled data. In practice, measurements are corrupted by noise, making derivative estimation unreliable.
In this section, we extend the framework to handle additive stochastic disturbances and estimate $\mat{P}_\lambda$---and thus the rank of the data-driven residual cross-moment---from cross-moments between the state and input signals, using only increments of $x$.
This is particularly natural in a stochastic setting, where $\dot x$ does not exist pointwise.
Assume that the measured state is an It\^o process satisfying the linear SDE
\begin{equation}\label{eq:hautus-ito-sde}
\begin{aligned}
  \d x(t)&=\big(\mat{A}x(t)+\mat{B}u(t)\big)\,\d t+\boldsymbol{\beta}\,\d W(t),\\
  x(0)&=x_0,
\end{aligned}
\end{equation}
where $W$ is a $q$-dimensional standard Brownian motion and $\boldsymbol{\beta}\in\R^{n\times q}$ is constant.
As before, define the stacked signal $z(t):=[x(t);u(t)]\in\R^{n+m}$ and the Hautus matrix
\[
  \mat{P}_\lambda := \begin{bmatrix}\mat{A}-\lambda \mat{I} & \mat{B}\end{bmatrix}\in\CC^{n\times(n+m)}.
\]
For $\lambda\in\CC$, define the It\^o residual differential
\begin{equation}\label{eq:hautus-ito-residual}
  \d m_\lambda(t):=\d x(t)-\lambda x(t)\,\d t=\mat{P}_\lambda z(t)\,\d t+\boldsymbol{\beta}\,\d W(t).
\end{equation}

\paragraph{Cross-moment factorization.}
Define the (random) stacked Gramian and cross-moment matrix
\begin{align}
  \label{eq:hautus-cross-moment-S}
  \widetilde{\mat{S}}_Z(T):=&\int_0^T z(t)z(t)^\top\,\d t\in\R^{(n+m)\times(n+m)},\\
  \label{eq:hautus-cross-moment-H}
  \widetilde{\mat{H}}_\lambda(T):=&
  \int_0^T \d m_\lambda(t)\,z(t)^\top
  =\mat{P}_\lambda \widetilde{\mat{S}}_Z(T)  +
  \boldsymbol{\beta}\int_0^T \d W(t)\,z(t)^\top\in\CC^{n\times(n+m)}.
\end{align}
The last term is a matrix-valued martingale with $\E[\int_0^T \d W(t)\,z(t)^\top]=0$, hence 
\[
\mathbb{E}\!\big[\widetilde{\mat{H}}_\lambda(T)-\mat{P}_\lambda\widetilde{\mat{S}}_Z(T)\big]=0.
\]
%
%
%
%
Let us estimate $\mat{P}_\lambda$.
Assume that $\widetilde{\mat{S}}_Z(T)$ is invertible and define
\begin{equation}\label{eq:hautus-cross-moment-P-hat}
  \widetilde{\mat{P}}_\lambda(T):=\widetilde{\mat{H}}_\lambda(T)\widetilde{\mat{S}}_Z(T)^{-1}.
\end{equation}
Then \eqref{eq:hautus-cross-moment-H} implies the exact error identity
\begin{equation}\label{eq:hautus-cross-moment-error}
  \widetilde{\mat{P}}_\lambda(T)-\mat{P}_\lambda
  =
  \boldsymbol{\beta}\left(\int_0^T \d W(t)\,z(t)^\top\right)\widetilde{\mat{S}}_Z(T)^{-1}.
\end{equation}
% Moreover, defining $\widetilde{\mat{G}}_\lambda(T):=\widetilde{\mat{P}}_\lambda(T)\widetilde{\mat{S}}_Z(T)\widetilde{\mat{P}}_\lambda(T)^*$, we obtain the fully data-dependent representation $\widetilde{\mat{G}}_\lambda(T)=\widetilde{\mat{H}}_\lambda(T)\widetilde{\mat{S}}_Z(T)^{-1}\widetilde{\mat{H}}_\lambda(T)^*$.
%
We want to characterize the statistical rate of convergence of $\widetilde{\mat{P}}_\lambda(T)$ to $\mat{P}_\lambda$ as $T\to\infty$.
Introduce the normalized quantities $\bar{\mat{S}}_Z(T):=\frac{1}{T}\widetilde{\mat{S}}_Z(T)$ and $\bar{\mat{H}}_\lambda(T):=\frac{1}{T}\widetilde{\mat{H}}_\lambda(T)$ so that $\widetilde{\mat{P}}_\lambda(T)=\bar{\mat{H}}_\lambda(T)\bar{\mat{S}}_Z(T)^{-1}$.
The It\^o isometry yields (componentwise) the bound
\begin{equation}\label{eq:hautus-cross-moment-ito-isometry}
  \E\left\|\frac{1}{T}\int_0^T \d W(t)\,z(t)^\top\right\|_F^2
  =
  \frac{q}{T^2}\E\left[\int_0^T \|z(t)\|_2^2\,\d t\right].
\end{equation}
If $\E[\int_0^T \|z(t)\|_2^2\,\d t]=\mathcal{O}(T)$ and $\|\bar{\mat{S}}_Z(T)^{-1}\|_2=\mathcal{O}_{\mathbb{P}}(1)$, then \eqref{eq:hautus-cross-moment-error} and \eqref{eq:hautus-cross-moment-ito-isometry} imply
\begin{equation}\label{eq:hautus-cross-moment-rate}
  \|\widetilde{\mat{P}}_\lambda(T)-\mat{P}_\lambda\|_2=\mathcal{O}_{\mathbb{P}}(T^{-1/2}).
\end{equation}
The $\mathcal{O}_{\mathbb{P}}(T^{-1/2})$ statement above can be strengthened to an explicit bound holding with probability $1-\delta$ and depending only on the (random) conditioning of $\bar{\mat{S}}_Z(T)$.
\begin{prop}[Cross-moment error bound]\label{prop:hautus-cross-moment-high-prob}
  Assume that $\widetilde{\mat{S}}_Z(T)\succ 0$. Then for every $\delta\in(0,1)$, with probability at least $1-\delta$,
  \begin{equation}\label{eq:hautus-cross-moment-high-prob}
    \|\widetilde{\mat{P}}_\lambda(T)-\mat{P}_\lambda\|_2
    \le
    \frac{\|\boldsymbol{\beta}\|_2}{\sqrt{T\,\sigma_{\min}(\bar{\mat{S}}_Z(T))}}
    \Big(\sqrt{q}+\sqrt{n+m}+\sqrt{2\log(1/\delta)}\Big) = \mathcal{O}(T^{-1/2}).
  \end{equation}
  Moreover, this implies the uniform bound $\sup_{\lambda\in\CC}\|\widetilde{\mat{P}}_\lambda(T)-\mat{P}_\lambda\|_2$ since the right-hand side of \eqref{eq:hautus-cross-moment-error} does not depend on $\lambda$.
Finally, by Weyl's inequality,
\[
  \big|\sigma_{\min}\big(\widetilde{\mat{P}}_\lambda(T)\big)-\sigma_{\min}(\mat{P}_\lambda)\big|
  \le
  \|\widetilde{\mat{P}}_\lambda(T)-\mat{P}_\lambda\|_2,
\]
\end{prop}
\begin{proof}
  See Appendix~\ref{app:proof:hautus-cross-moment-high-prob}.
\end{proof}

\begin{rmk}
The above result provides statistical guarantees for the Hautus test. In particular,
\begin{align*}
\sigma_{\min}(\mat{P}_\lambda)
&\ge \sigma_{\min}\!\big(\widetilde{\mat{P}}_\lambda(T)\big)
 - \big|\sigma_{\min}(\mat{P}_\lambda) - \sigma_{\min}\!\big(\widetilde{\mat{P}}_\lambda(T)\big)\big| \\
&\ge \sigma_{\min}\!\big(\widetilde{\mat{P}}_\lambda(T)\big) - \|\widetilde{\mat{P}}_\lambda(T) - \mat{P}_\lambda\|_2 \\
&\ge \sigma_{\min}\!\big(\widetilde{\mat{P}}_\lambda(T)\big) -  \mathcal{O}(T^{-1/2}).
\end{align*}
\end{rmk}

\subsection{A Fourier-domain approximation without derivatives.}
The cross-moment in \eqref{eq:hautus-cross-moment-H} can also be approximated in the frequency domain without forming $\dot x$.
For each $\omega\in\R$, define the Fourier transform of the residual increment
\begin{equation*}
  \widehat m_\lambda(\ii\omega)
  :=
  \int_0^T e^{-\ii\omega t}\,\d m_\lambda(t)
  =
  \int_0^T e^{-\ii\omega t}\,\d x(t)-\lambda\widehat x(\ii\omega),
  \qquad
  \widehat x(\ii\omega):=\int_0^T x(t)e^{-\ii\omega t}\,\d t.
\end{equation*}
Since $t\mapsto e^{-\ii\omega t}$ is of bounded variation, It\^o integration by parts yields
\[
  \int_0^T e^{-\ii\omega t}\,\d x(t)
  =
  x(T)e^{-\ii\omega T}-x(0)+\ii\omega\int_0^T x(t)e^{-\ii\omega t}\,\d t,
\]
hence
\begin{equation}\label{eq:hautus-fourier-y}
  \widehat m_\lambda(\ii\omega)
  =
  x(T)e^{-\ii\omega T}-x(0)+(\ii\omega-\lambda)\widehat x(\ii\omega).
\end{equation}
Thus, $\widehat m_\lambda(\ii\omega)$ can be computed from an FFT of $x$ plus the boundary terms $x(0),x(T)$.
Define also $\widehat z(\ii\omega):=\int_0^T z(t)e^{-\ii\omega t}\,\d t$.
If $x$ is absolutely continuous so that $\d m_\lambda(t)=(\dot x(t)-\lambda x(t))\,\d t$, then Parseval gives the deterministic identities
\[
  \mat{H}_\lambda(T)=\int_{\R}\widehat m_\lambda(\ii\omega)\widehat z(\ii\omega)^*\,\frac{\d\omega}{2\pi},
  \qquad
  \mat{S}_Z(T)=\int_{\R}\widehat z(\ii\omega)\widehat z(\ii\omega)^*\,\frac{\d\omega}{2\pi}.
\]
In the It\^o setting, the unwindowed energy $\int_{\R}\|\widehat m_\lambda(\ii\omega)\|_2^2\,\d\omega$ is not finite, so frequency-domain computations should be interpreted with an explicit cutoff/windowing when needed.
